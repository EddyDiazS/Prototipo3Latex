\documentclass[conference]{IEEEtran}

% Language setting
% Replace `english' with e.g. `spanish' to change the document language
\usepackage[spanish]{babel}

% Useful packages
\usepackage{amsmath}
\usepackage{graphicx}
\usepackage{url}

% Title and author info
\title{Dashboard de Ventas con Django y Plotly, PROTOTIPO 3}
\author{
\IEEEauthorblockN{Ángel Thomas Rodríguez Pinto - 506221014\\ angelt.rodriguezp@konradlorenz.edu.co}
\and
\IEEEauthorblockN{Eddy Andrés Díaz Santos - 506221024\\ eddya.diazs@konradlorenz.edu.co}
}


\begin{document}
\maketitle

\begin{abstract}

This report presents a detailed analysis of the inventory and sales management process carried out between 2020 and 2024, using tools such as Django, Plotly, Python, HTML and CSS. The project was divided into several stages, starting with the creation of the web application using Django and the integration of Plotly for data visualization. During the development, an intuitive interface was designed that allowed the exploration and understanding of sales data in an efficient way with the help of HTML, Python and CSS. The organized structure of the project facilitated collaborative development and the implementation of security measures to protect confidential information. The results obtained were an effective tool for sales analysis, providing interactive graphics and a complete management interface.


\end{abstract}

\section{Introduction}

En este informe se presenta un análisis detallado del proceso de gestión de inventario con respecto a un sistemas de ventas, teniendo en cuenta mes, año y ubicacion por barrio, llevado a cabo durante el período comprendido entre los años 2020 y 2024. Durante este tiempo, se emplearon diversas herramientas y tecnologías, incluyendo la plataforma Django para el desarrollo web y la biblioteca Plotly para la visualización de datos, entre otras mas. El informe examina las actividades realizadas, resaltando los métodos utilizados, los desafíos enfrentados y los resultados obtenidos en la gestión del inventario.



\section{Creacion}

\subsection{Vista general}

La creación del sistema de gestión de inventario fue un procesocon varias etapas en las cuales se aprovechó de diversas tecnologías y herramientas. Utilizando la terminal como punto de partida, se inició un proyecto Django, un framework de desarrollo web en Python, para construir la aplicación. La combinación de Django con HTML y CSS permitió diseñar una interfaz de usuario atractiva y funcional para la gestión de ventas.

Durante el desarrollo, se integraron bibliotecas como Plotly, especializada en visualización de datos, para proporcionar gráficos interactivos y tableros de control que facilitaran el análisis del inventario y las tendencias de ventas. Esta integración permitió una mejor comprensión de los datos y una toma de decisiones más detallada.

Se implementaron medidas de seguridad en el sistema para proteger la información confidencial del inventario y garantizar el cumplimiento de las políticas de acceso. Se configuró un servidor local para probar y desplegar la aplicación de forma segura antes de su implementación en un entorno de producción.

La creación del sistema de gestión de inventario fue un proceso colaborativo que aprovechó herramientas como Django, Plotly, HTML, CSS y Python, junto con la terminal y un servidor local, para desarrollar una solución completa y efectiva para la administración del inventario de un sistema de ventas inicial.

\subsection{Desarrollo}

Para el desarrollo del prototipo se uso la base realizada en clase, en la cual hicimos un sistema de inventarios que era nos daba una entrada a lo que seria posteriormente un sistemas de control de ventas, teniendo en cuenta que la base suministrada en un principio era algo basico que mas adelante se pudo convertir en gestor de ventas mucho mas completo. 
En el desarrollo se tuvo en cuenta como podria ser subdividido el gestor Django, para posteriormente con ayuda de la biblioteca Plotly, mostrar los datos de manera grafica. Utilizando Django y Python, se diseñó un sistema para recopilar, procesar y visualizar datos de ventas. Se implementaron modelos de base de datos para almacenar información detallada sobre las de ventas, incluyendo el barrio, la cantidad vendida y el mes correspondiente.

Como se menciono anteriormente se uso la biblioteca Plotly para generar gráficos interactivos que permitieran a los usuarios explorar fácilmente las tendencias de ventas a lo largo del tiempo. Estos gráficos proporcionaron una visión clara y concisa de los patrones de ventas, lo que facilitó la identificación de oportunidades y desafíos comerciales.


\subsection{Division y uso de la interfaz }

La interfaz se dividió en varias secciones para una mejor organización y presentación de la información. En la parte superior, se incluyó un título destacado que resaltaba la funcionalidad principal del módulo, seguido de una tabla que mostraba los detalles específicos de las ventas, como el barrio, el mes y la cantidad vendida. Esta tabla proporcionaba una vista detallada de los datos tabulados.

Se incorporaron gráficos interactivos generados con la biblioteca Plotly. Estos gráficos presentaban visualmente las tendencias de ventas a lo largo del tiempo y en diferentes ubicaciones, lo que facilitaba la comprensión de las ventas. Los usuarios podían explorar los gráficos de forma dinámica, aplicando zoom, desplazándose y filtrando según sus necesidades específicas.

Finalmente, en la parte inferior de la interfaz, se incluyeron botones de navegación que permitían a los usuarios acceder fácilmente a otras partes del sistema, como la administración de ventas y el panel de control. Estos botones proporcionaban una manera rápida y conveniente de cambiar entre las diferentes funciones del sistema, mejorando así la experiencia del usuario y la eficiencia operativa. En conjunto, esta división de la interfaz proporcionó una experiencia intuitiva y completa para el análisis y la gestión de ventas.

\section{Resultados}

La implementación del gestor de ventas resulto en una herramienta efectiva para gestionar y visualizar datos de ventas de manera intuitiva. Se utilizó Django, Plotly, Python, HTML y CSS para desarrollar una aplicación web que proporciona análisis detallados y gráficos interactivos de las ventas. La estructura organizada del proyecto facilitó el desarrollo y la colaboración, mientras que la interfaz de usuario intuitiva mejora la experiencia del usuario al navegar y comprender la información de manera eficiente, hay que tener en cuenta que mediante el proceso de creacion se tuvieron ciertas dificultades las cuales se iban resolviendo poco a poco, para una mejor experiencia de usuario o administrador, ya que es otro factor a tener en cuenta se realizo una interfaz de administrador para llevar el control completo de las ventas. En resumen, el proyecto logró su objetivo de proporcionar una solución útil y fácil de usar para el análisis de ventas.

\section{Conclusiones}

La implementación del gestor de ventas demuestra la versatilidad de las herramientas utilizadas, como Django, Plotly y Python, para desarrollar aplicaciones web interactivas y visualmente atractivas. La capacidad de generar gráficos dinámicos y personalizados ha mejorado significativamente la comprensión y la toma de decisiones basadas en datos para la gestión de ventas, no solamente esto si no tambien muestra el alcanze que se puede obtener con ciertas herramientas de desarrollo. 

\bibliography{sample}

\end{document}

